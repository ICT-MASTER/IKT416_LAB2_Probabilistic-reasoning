\documentclass{article}
\usepackage[english]{babel}
\usepackage[utf8]{inputenc}
\usepackage{fancyhdr}
\usepackage{natbib}
\usepackage{graphicx}
\usepackage[]{algorithm2e}
\usepackage[procnames]{listings}
\usepackage{color}
\usepackage{tikz}
\usetikzlibrary{arrows}
\usetikzlibrary{calc,through,backgrounds}
\usepackage{pgfplots}
\pgfplotsset{compat=1.8}

\usepackage{tcolorbox}





\pagestyle{fancy}
\fancyhf{}

%% Header %%
\rhead{Learning Systems | IKT440}
\lhead{Mandatory Exercise 2}
\rfoot{Page \thepage}


%% Front-page %%
\title{Probalistic Reasoning \\  \large Mandatory exercise 2}
\author{Christian Kråkevik \and Per-Arne Andersen}
\date{September 2015}


\begin{document}

%% PYTHON HILIGHT %%
\definecolor{keywords}{RGB}{255,0,90}
\definecolor{comments}{RGB}{0,0,113}
\definecolor{red}{RGB}{160,0,0}
\definecolor{green}{RGB}{0,150,0}

\lstset{language=Python,
        basicstyle=\ttfamily\small,
        keywordstyle=\color{keywords},
        commentstyle=\color{comments},
        stringstyle=\color{red},
        showstringspaces=false,
        identifierstyle=\color{green},
        procnamekeys={def,class}}


\maketitle

\newpage

\section{Implementation}
We decided to implement the solution in Python. Our first goal was to solve the problem in a pseudo-like approach. After some tinkering, our purposed solution looked like this :
\\

\begin{algorithm}[H]
  \Comment{< # A string with all words - stripped out metadata.>}
  readAllWords(numOfTraining)\;
  \Comment{< # A dictionary containing all words>}
  constructReferenceDict\;

 \For{numOfCategories}{
  loadAllSubTrainingData()\;
   \For{numOfWordsInRefDict}{

    countOccurencesOfWord(x)\;
    saveProbOfWord(x)\;
 }
 saveSubListDict()\;

 }
 \caption{Creating dictionaries}
\end{algorithm}

\vspace*{8px}

\begin{algorithm}[H]


  readAllWords(randomArticle)\;
  p-old = 0\;
  p = 0\;

 \For{allSubCat}{
     \For{wordInArticle}{
        p + = log(subCat[wordInArticle])\;
     }
     \eIf{p bigger p-old}{
      p=p-old\;
    }{
    p-old=p\;
   }
 }
 print(p)
 \caption{Classifying}
\end{algorithm}

\vspace*{8px}
\newpage
\subsection {Dictionaries}
We read all the words from all the texts to create a referenceDictionary. This dictionary is used as a basis.
The next step is to create a dictionary for each sub-category. This dictionary should consist of a word/p-for-word
design, so that we can check a given word for the percentage for a given subcategory.

\newline
\noindent We achieve this by loading in all the words from all the training articles for a given category. We then strip it of
all metadata, and identifying tags like email, numbers and so on. The next step is to iterate trough each word in our referenceDict.
For each of the words, we count the amount of times this word exists in the list of sub-cat words. We use this variable as a basis
for calculating the percentage for THIS word existing in THIS subcategory.

\begin {lstlisting}
pForWord = (nForWord+1)/(len(listOfWord)+len(referenceDict))
\end{lstlisting}

\newline

\subsection {Classification}
The classification is the process where we calculate the correct subclass for a given text. In our task we first loads all the
subclasses from .json serializable objects. We now have 20 sub-category dictionaries with the probabillity for all the words. We
then continue to read every single word of the article. The next step is to iterate trough all the words in the given text, and append
the P value for that word to get a result. After we have done this for all the words, we move to the next subcategory. The subcategory
with the highest P, is the one that is chosen as the most likely category.

\subsection {Improvements}
The generation of subcategory dictionaries was slow, so we decided to implement a multi-threaded solution. This was achieved by supplying
chuncks of words to a single thread. Thus enabling us to do 8 threads and utilizing all the power from our CPU.

\newline
\noindent We also implemented a multi-threaded solution for the classification process, with 8
different threads, enabling us to calculate 8 articles at the same time.  This worked partially, but after realizing
that dictionaries was superior in speed as opposed to lists, we achieved a generation time of 13s instead of 5 hours.



\subsection {Results}
With a subset of 900 learning articles, times 20 categories. We achieved a percentage of 78.8234.

\end{document}
